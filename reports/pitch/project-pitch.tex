%\documentclass{neu_handout}
\documentclass[a4paper, 11pt]{article}

\usepackage{url}
\usepackage{amssymb}


\usepackage{comment}  
\usepackage{fullpage} % changes the margin
\usepackage[margin=0.45in]{geometry}
\pagenumbering{gobble}          
\usepackage{hyperref}

\begin{document}


\title{DS5230 Project Pitch : Pattern Recognition in Accidents in the UK}


\author{
  Sahasrabhojanee, Adwait\ \\     \texttt{sahasrabhojanee.a@husky.neu.edu}
  \and
  Sreekumar, Sreejith\  \\ \texttt{sreekumar.s@husky.neu.edu}
  \and
  Yu, Xue \\  \texttt{yu.xue1@husky.neu.edu}
  \and
  Li, Xuexian  \\ \texttt{li.xuex@husky.neu.edu}
}
\maketitle
\section{Abstract}
In September 2017, Kaggle released a dataset that contains an aggregated collection of 1.6 million accident records compiled from a few UK government sources in the area during the time span of 2000 - 2016. For the course project of Unsupervised Machine Learning and Data Mining - DS 5230, we will be analyzing this data, working towards uncovering patterns in accidents. \\

   Our final objective would be to derive insights and provide useful information to the managerial personnel for the betterment of traffic management plans. \\

   \section{The Dataset}
   The dataset contains two types of files:

   \begin{itemize}
     \item The average annual daily flow of traffic on major roads in the U.K
     \item Records on over 1.6 million road accidents, spread across three files
   \end{itemize}
   Attributes related to the accident are provided as columns in the records. These include the date and location of the accident, the types of the road and the vehicle involoved, the number of casualties, and the weather and geographic conditions. A complete description of the fields has been provided at : \url{https://www.kaggle.com/daveianhickey/2000-16-traffic-flow-england-scotland-wales}
   
   \section{Questions we intend to answer}
   During the course of the analysis, we will be focusing on answering the following questions:
   \begin{enumerate}
     \item Does the season/time affect the number of accidents?
     \item Does the type of the road significantly affect the chance of an accident occurring on it?
     \item What determines whether or not a police officer will attend the scene of the accident?
     \item Are there factors other than those recorded in the data that cause a higher number of accidents in a particular area? (Population, for instance)
     \item Are any of the factors linearly related to each other?
   \end{enumerate}

   \section{Methods}
   
   We intend to use clustering methods to answer questions 1 through 3, holding out a potentially important variable (which takes one of K values) and group the data into K clusters. We will then compare the clusters formed with the held-out variable to see if they are roughly similar. \\
   For analyzing the factors mentioned in question 4, we intend to cluster the data, and then try to explore external factors which match the clusters formed. \\
   Exploratory data analysis and statistical studies will be conducted on the variables to understand correlations and answer question five.  
   

   \section{Results}
   \begin{enumerate}
   \item Visual temporal and spatial exploratory data analysis of accident data
   \item Answers to the aforementioned questions
   \item  Outputs from the statistical analysis of variables
   \end{enumerate}

     \section{Work Division}
     In a subgroup of two, we will be working towards answering questions one through four (two questions for each subgroup). Question five, being more generic, will
     be a split across everyone. This will proceed in a series of group discussions and experiments as the project progresses.
     

     \section{References}
     \begin{itemize}
       \item Kaggle, 2017. 1.6M accidents \& traffic flow over 16 years. Accessed at October 10, 2017. \url{https://www.kaggle.com/daveianhickey/2000-16-traffic-flow-england-scotland-wales}
     \end{itemize}
\end{document}
